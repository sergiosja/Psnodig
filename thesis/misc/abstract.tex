\chapter*{Abstract}

Pseudocode is commonly used to provide a description of an algorithm at a suitable level of abstraction. Pseudocode is technically void of any syntactic rules, allowing us to omit or de-emphasise certain aspects of our programs. However, with this amount of freedom, it is easy to accidentally misplace parantheses, fail to consistently rename variables, or even demonstrate wrong formulas. Additionally, we are now tasked with maintaining two versions: the executable source code and its corresponding pseudocode. \\

In this thesis, we present \textbf{Psnodig}, a transpiler that converts source code to pseudocode and flowcharts. This way, we are able to write our programs in an imperative, high-level language, test them, and when satisfied, change their abstraction levels for presentation. \\

Psnodig is designed to be \textbf{extensible}, allowing the addition of parsers and code generators; \textbf{executable}, running the source code we write; and \textbf{presentable}, automatically converting said source code to a presentation-only version. To evaluate these properties, selected algorithms from a textbook on algorithm theory were chosen as a case study. \\

The results indicate that Psnodig achieves these goals: parsing programs, running their code with satisfactory results, and transforming them further to various levels of abstraction.