\chapter{Discussion}

In this chapter, we will mainly discuss the results we obtained in Section 6. We will first reflect on the results, before discussing the current limitations of Psnodig.

\section{Reflections}

Based on the results from Section 6, we believe that Psnodig successfully meets the goals set in Section 1.2, and can provide an answer to the problem definition introduced in Section 3.1.

\subsection{Extensible}

Currently Psnodig only has one parser, for an imperative, C-like language. As we saw in Section 6.1, the parser successfully converted programs of varying sizes to Psnodig ASTs. These ASTs were later successfully executed, and also losslessly converted back to the original source programs, to Python programs, and presentation-only targets. \\

The parser consists of 328 lines of code, including boilerplate code like library imports, and putting all exported functions on their own separate line. 28 of these lines are function signatures. \\

The Gourmet writer is even more succinct, consisting of only 224 lines of code, including boilerplate code. The Python writer is slightly larger, with 300 lines of code. However, we could draw parallels between the required amount of code to write a parser and a writer for the same language. In that case, it should be highly achievable to add a parser for (at least a subset of) Python.

\subsection{Executable}

The results from the evaluation were very satisfying with regards to executability. All algorithms performed precisely as expected, despite being tested on very different inputs, as well as producing different outputs. \\

The search functions both returned True when they found the target number in the list, and returned False in the case where it was not present. \\

The BST functions all returned the expected tree values. FindMax successfully located the node with the highest value. Insert was able to insert both integer and decimal values into the tree, whilst preserving the tree's properties. The same applies to Remove, that was able to remove nodes with one, two, and no children, correctly replacing the appropriate values and discarding the target node. Lastly, Search found all the nodes in the tree, and let us know when we looked for a node that was not present. \\

The sorting algorithms gave us the exact same outputs: three sorted lists. The first test shows that they are able to handle empty lists, avoiding OutOfBounds-exceptions during computation. The second test shows that they are indeed capable of sorting lists of random integers. The third tests ensures that when data is already ordered, it stays unaltered. \\

Both graph algorithms managed to successfully retrieve all the nodes where there was a valid path. Since we also printed the node they were currently visiting, we could see that the algorithms acted differently, albeit expectedly. BFS explored the graph one depth level at a time, exploring node 1 and 4 after starting on node 3. DFS, however, explored one path as far as it could. Despite also starting on node 3, it explored 4 and then 5, before looking at 1.

\subsection{Presentable}

Both of our presentation-only writers were able to convert Psnodig ASTs to LaTeX targets, that we then compiled and presented. \\

The TBP writer successfully produced pseudocode from Psnodig ASTs. It presents the programs in a similar fashion to the original source program, but is able to abstract over details like swapping variables, finding pivots and associating lists with keys. The possibility of defining our input and output allows us to give the reader a more predictable journey when reading the pseudocode. \\

The IBP writer successfully produced flowcharts, though we are not satisfied with all of the results. However, considering it converts ASTs deterministically, and uses very few lines of code, we believe it to be at least a decent prototype. It especially excels when dealing without control flow statements, with isolated control flow statements, and with if-statements containing many clauses.

\section{Limitations}

Section 6 also identified some limitations of Psnodig. While all the goals in Section 1.2 are met, certain aspects leave us with room for improvement. Discussing these limitations will lead to a more predictable developer experience, since we will be able to anticipate these potential drawbacks.

\subsection{Extensible}

Psnodig is primarily built to work with imperative languages, like Go and Python. This made it comfortable to write a parser for Gourmet, as well as the writers for Gourmet and Python. However, adding a parser for a language from a different paradigm might be more challenging. \\

To add a parser or writer to Psnodig in the first place, they must be written in Haskell.\footnote{Technically, they can be written in a different language and \textit{transpiled} to Haskell.} Even though we believe it is a good fit for this type of job, the language itself is considered to be a difficult programming language to learn~\cite{haskellIsHard1}.

\subsection{Executable}

Naturally, we would like all programs to be executable. However, the interpreter currently stands without an implementation for the statement constructors \textbf{break} and \textbf{continue}. If a program flow runs into either of them, the program will terminate with a warning. Except for these, all Psnodig data types are handled by the interpreter. \\

The way execution currently works, is that the interpreter runs the program, and returns either a final state or a runtime error. When we use the print function in our programs, their arguments are added to the final state. This means that we only see the resulting print statements after the program has run its course. \\

The biggest downside of this, is that we have no way to really identify the bottleneck of infinite loops. Where you can do something like \texttt{while True \{ print("loop") \}} to print \texttt{loop} infinitely many times in many other languages, Psnodig will never print \texttt{loop}, because the program never terminates.

\subsection{Presentable}

As mentioned in \Cref{5_flowchartWriter}, we keep track of a list of ids. This is to help statements point back to their original conditions, and also to know where to continue after writing the inner statements of a control flow statement. However, it does not always seem that this list is being updated correctly. \\

As such, we can encounter problems when we convert nested control flow statements (\texttt{While}, \texttt{For}, \texttt{ForEach}, and \texttt{If}) to IBP. For instance, it makes statement nodes crash more often than necessary, like we see in \Cref{Flowchart of Bubble sort implementation.}. In \Cref{Flowchart of Bucket sort implementation.} we can also see that edges sometime interfere with nodes, making it a bit difficult to know where they actually point. \\