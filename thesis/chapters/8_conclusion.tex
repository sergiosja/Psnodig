\chapter{Conclusion}

\forsup{Important: Remind the reader of all the good stuff!}

This section will give a summary of our contributions and discuss future work, before concluding the thesis.

\section{Future Work}

make it more customisable. ref
\begin{verbatim}
    -- hadde vært kult om man kunne sende med egne sånne? feks gjennom en fil, også er brukeren selv ansvarlig for at alt er riktig
-- istedenfor å være bundet til startstop, io osv. å legge inn en egen \tikzstyle{sergey_custom} = [rectangle, minimum width=15cm, ..]
\end{verbatim}
fra flowchart writeren

\subsection{Psnodig}

The syntax of Psnodig contains all the necessary building blocks to write complex computer programs, but as previously mentioned, the syntax shows the maximum available. Parsers and writers can easily utilise as little of the syntax as they want, for instance if we wish to create a simple calculator language. \\

Therefore, it could be interesting to expand Psnodig's syntax to include more things we commonly see today in programming, like lambda functions and more data structures. This would make the language even more flexible, and potentially make it more tempting for people to use.

\forsup{add more parsers!!!}

\subsection{Interpreter}

The interpreter has its primary focus on correctness, and is not particularly optimised for speed. There are parts of the interpreter that could be optimised, utilising Haskell's features to a greater extent. For one, variable scopes are lists of \texttt{(String, Value)}-pairs, but the innermost layer could have been a mapping, which would shrink the ammortised runtime of lookups from $O(n)$ to $O(1)$. \\

Another flaw is that the interpreter will only stop on error or success. For instance, We do not have a way of identifying infinite loops by print statements like we can in many other language. This is because the print statements appear on the screen only when the program has terminated.

\subsection{Writers}

The writers all work in their own right, but there are always improvements to be made. This section covers what we believe to be the ones to make.

\subsubsection{Python Writer}

\forsup{Currently, Python is a subset of Python, but we are not sure to what extent we cover the standard Python language. There are things we do not touch upon, like list comprehension, though there are patterns in the Psnodig syntax that could be recognised and converted accordingly.}

\forsup{no we dont!! we dont gaf. psnodig is first and foremost made 4 tbp and ibp :)}

\subsubsection{Pseudocode Writer}

We have utilised the Algorithm2e package, but there might be a different package we could use.

\subsubsection{Flowchart Writer}

As discussed in Section 7.2.1, the IBP writer struggles with complex flowcharts where we nest the statements types \texttt{While}, \texttt{For}, \texttt{ForEach}, and \texttt{If}. This could be improved by making the edges longer in these cases, or choosing different placement rather than always building flowcharts downwards.

\section{Summary of Contributions}

We introduced an imperative C-like programming language Gourmet. \\

We designed a DSL Psnodig, and created an interpreter that runs on its internal representation. \\

We have added four writers to Psnodig, two of them being executable programming languages, and two of them being presentation-only targets.