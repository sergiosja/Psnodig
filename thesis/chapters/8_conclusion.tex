\chapter{Conclusion}

This section will give discuss future work before concluding the thesis, finishing with a summary of our contributions.

\section{Future Work}

Despite being fairly satisfied with the results of our work, there are aspects we wish we had time to explore further. This section will touch upon some of these.

\subsection{Expanding Psnodig's Core}

The syntax of Psnodig contains all the necessary building blocks to write complex computer programs, but as previously mentioned, the syntax shows the maximum available. Parsers and writers can easily utilise as little of the syntax as they want, for instance if we wish to create a simple calculator language. \\

Therefore, it could be interesting to expand Psnodig's syntax to include more things we commonly see today in programming, like lambda functions and priority queues. This would make the language even more flexible, and potentially more accessible. \\

Given that Psnodig comes with an interpreter, it would be exciting to expand its list of built-in functions with more modern utilities like string interpolation and swapping elements in a list. This could also help to make the tool become more desirable.

\subsection{Making Writers Parameterisable}

Despite allowing for many more macros than we use, neither the TBP- nor the IBP writer is currently parameterisable. To adjust their configurations at all we are forced to change their implementations directly. \\

It would be very interesting to explore a way to make the writers parameterisable. For instance, to change the node styles we currently apply for our IBP writer, or to change the \texttt{\textbackslash SetKwProg\{Prog\}\{Title\}\{is\}\\\{end\}} macro for the TBP writer. \\

We believe that users will feel more ownership to Psnodig if they are able to leave their mark own on it. However, they should not have to write an entirely new writer just because they would like to change style attributes like fonts and colours.

\subsection{Exploring More Packages}

Despite our general satisfaction with the outcome of Psnodig, we never went this far implementation-wise with any other LaTeX package. There are multiple alternatives available, like \texttt{algpseudocode} for TBP and \texttt{Graphviz} for IBP. It would be very interesting to add writers using these libraries and compare the results with our current ones. \\

\section{Concluding Our Work}

In this thesis, we introduced Psnodig, a general transpiler with options for pseudocode and flowcharts. The motivation for this thesis was to create a tool that lets us combine the benefits of both executable and non-executable formats. At the same time, to make it more flexible, our tool should be able to support multiple programming languages. We condensed our aim into three points: Psnodig should be extensible, executable, and presentable. \\

The results of this thesis show that Psnodig successfully accomplishes these goals. We created a parser for a C-like language that we coined Gourmet, and writers for Gourmet itself, Python, pseudocode and flowcharts. They were unit tested individually, making sure that they were able to convert isolated parts to and from the internal representation of Psnodig. We used Psnodig to transpile programs of varying complexity, and we were able to test each one with satisfactory results in the process. \\

The tool is not perfect, however, and it is particluarly showing in the flowcharts when we transpile complex programs. Additionally, it only has one parser for writing source code, which might limit its accessibility. This is a start, and with more work in the future, it could potentially become very useful. \\

To succinctly summerise the contributions of this thesis, we have introduced a transpiler Psnodig, that is extensible: we can add parsers and writers at will; executable: we can run programs that are successfully parsed to an internal representation of Psnodig; and presentable: the code we write can be converted to presentation-only formats, enabling us to showcase its underlying logic with a different level of abstraction.
