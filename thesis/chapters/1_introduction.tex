\chapter{Introduction}

Pseudocode is commonly used to provide a description of an algorithm at a suitable level of abstraction. It is meant to work as a comprimise between a low-level implementation in a specific programming language, and a natural language description of a problem solution~\cite{whatIsPseudocode}. \\

An advantage of pseudocode is the lack of standardisation, therefore authors are not tied down to the syntax of any particular programming language. This gives them complete freedom to omit or de-emphasize certain aspects of their algorithms. Consequently, pseudocode is first and foremost aimed to serve as a tool for presentation. \\

Consequently, as pseudocode is not executable, there is no omniscient way to verify its correctness. This can, in turn, lead to accidental inclusion of critical inaccuracies. % especially when working with algorithms one is less acquainted with.
When we write code in IDEs, programming languages are often accompanied by static analysis tools that detect anti-patterns and warn about bad practices~\cite{manyLinters, whatIsALinter}. Psudocode writers, on the other hand, are left to their own devices, as identification of anti-patterns and bad practices generally relies on the existence of established standards. \\

\forsup{- Bør jeg forklare hva en IDE er?}

\section{Motivation}

Correct presentations are important in education where the goal is to teach students concepts they were previously unfamiliar with. At university level, concepts within algorithms and data structures have traditionally proved challenging for undergraduates~\cite{algorithmsAreHard1, algorithmsAreHard2}. If their first impression of an algorithm is an incorrect presentation, their path is already hampered. \\

In this thesis we present a tool called \textbf{Psnodig} (pronounced snoo-dee~\footnote{Imagine the \textbf{sn} in \textbfit{snow}, the \textbf{oo} in \textbfit{cool}, and the \textbf{dee} in \textbfit{deep}. The silent \textbf{p} is an ode to the silent p in the word \textbfit{pseudocode}, naturally.}), which allows us to translate source code to other, perhaps less technical presentations. The presentation targets in this thesis are pseudocode and flowcharts. The target audience is people who study or work with algorithms. \\

The positive effect of teaching algorithms with flowcharts as an alternative to traditional code has been researched since at least the 1980's, and is still being researched this decade~\cite{flowchartsAreGood1, flowchartsAreGood2, flowchartsAreGood3}. However, tools for direct translation from source code to flowcharts does not seem to be widespread. \\

We spend much more time \textit{reading} code than we do writing it~\cite[14]{weReadMoreThanWeCode}. Tools like IDEs and linters can only help us so much when it is the \textit{logic} of our programs that we fail to grasp. We believe that Psnodig can be a valuable tool for authors who want to present their algorithms, as well as students who wish to get a better understanding of them. \\

\forsup{- Er det noe poeng i å referere til bestemt sidetall?}

We aim to promote algorithmic thinking through various forms of representation. We believe that this can aid us to better understand and modify our code, which in turn can lead to more efficient and effective programming practices. By not having to worry about syntactic intricacies, the audience can focus entirely on logic underlying the algorithms. \\

Another key point is that the softwares that \textit{do} let us translate source code to other representations, are completely independent of each other. Most of them also operate with their own DSL to write source code, which makes them even more unavailable. Psnodig, on the other hand, attempts to centralise all these resources, making it a powerful all-in-one tool for any tranlation of our wishes. \\

\forsup{- Bør jeg forklare hva DSL er?}

\forsup{- Er dette paragrafet for en for ``mystisk'' måte å avslutte subsectionen på? Ettersom jeg ikke forklarer hva jeg mener}

\section{Goals}

The overall goal of this thesis is to construct a tool with the following properties:

\begin{itemize}
    \item \textbf{Presentable}, the user can transition their source code to a different of abstraction.
    \item \textbf{Extensible}, the user can add parsers and code generators to work with Psnodig.
    \item \textbf{Customisable}, the user can manually alter the final result.
    \item \textbf{Executable}, the user can run the source code they have written.
\end{itemize}

\forsup{Dette er første gangen jeg nevner `parser` og `code generator`. Bør jeg referere til Section 2.\_.\_, siden jeg ikke forklarer hva det er?}

% Standard tools traditionally fulfill one or the other. Programming languages, in which you can write programs and test them, do not necessary have the appropriate abstraction level to be understood by students at all levels. Pseudocode, on the other hand, is intentionally not executable, and thus the presented ideas cannot be tested directly. The perk of centralising the resources to a single tool makes the job easier for everyone involved. \\

In chapter 3 we will analyse some tools that perform well in one or more area, but fail to satisfy all four.

\section{Contributions}

The main contribution of this thesis is the Psnodig tool, which facilitates the translation of executable source code to a number of different variants. This is indented to give people an easy and accessible way to look at their code from a different perspective. By using the Psnodig tool, we aim to help people to spend more time writing meaningful code and less time mastering LaTeX libraries, writing boilerplate code, and worrying about maintaining multiple sources. \\

The Psnodig tool comes with a parser for a simple imperative language we call Gourmet, to serve as a proof of concept. It is also accompanied by multiple code generators, which are able to transform a Psnodig abstract syntax tree (AST) back to Gourmet source code, as well as pseudocode and flowcharts written in LaTeX. The latter two utilise the Algorithm2e and TikZ packages, respectively. \\

To summarise, the contributions include:
\begin{itemize}
    \item Psnodig, a tool for translating code from one representation to another. It also comes with an interpreter which works on the intermediate AST representation, so that we can run our code.
    \item The Gourmet programming language, mainly inspired by Go and Python, as a proof of concept. This includes a parser for converting tokens to an AST, as well as a writer to convert the AST back to Gourmet code.
    \item A writer for presenting ASTs with text based pseudocode, utilising the Algorithm2e package in LaTeX.
    \item A writer for presenting ASTs with flowcharts, utilising the TikZ package, also in LaTeX.
\end{itemize}

\section{Project Source Code}

All the source code from the master thesis can be found on Github.~\footnote{The link is \url{https://github.uio.no/sergeyj/Master}. It is actually divided into two folders: \textbfit{psnodig}, containing the source code for the Psnodig tool, and \textbfit{thesis}, containing the LaTeX source code for this thesis.} (NOTE: Alt ligger fortsatt på uio-githuben. Jeg må få overført det til github.com på et tidspunkt. :smilefjes:).