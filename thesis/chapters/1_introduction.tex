\chapter{Introduction}

Pseudocode is commonly used for providing a description of an algorithm at a suitable level of abstraction. It is meant to be a comprimise between a low-level implementation in a specific programming language, and a word-for-word solution to problems in a natural language. \hfill \\

% However, a disadvantage of no standardisation is the subsequent lack of a maintained tool ... liker veldig godt ordet "subsequent" hehe. skulle ønske jeg kunne bruke det her et sted

An advantage with pseudocode is the lack of standardisation, therefore authors are not tied down to the syntax of any particular programming language. This gives them complete freedom to omit or de-emphasize certain aspects of her algorithms. Thus, pseudocode is used exclusively for presentation. However, as pseudocode is non-executable, there is no omniscient way of verifying its correctness. This can, in turn, lead to accidental inclusion of critical inaccuracies, particularly when working with lesser-known algorithms. \hfill \\

\section{Motivation}

Correct presentations are especially important in academia, where the goal is to teach students concepts they were previously unfamiliar with. Traditionally, concepts within the algorithms and data structures realm have proved challenging for undergraduates \cite{10.1145/2157136.2157148}. If then their first impression of an algorithm is an incorrect presentation, their path is already hampered. \hfill \\

In this thesis, we present a tool, "Psnodig", which allows for transpiling executable code to other, perhaps less technical presentations. The presentation targets in this thesis are pseudocode and flowcharts. The target audience is students who are already familiar with some C-like programming language syntax.

The reader might be familiar with the concept of \textit{compiling}. Traditionally, this is a process where a \textit{compiler} reads a program in a high-level language, e.g. Go, and translates it to an executable target program \cite{DBLP:books/aw/AhoSU86}. \hfill \\

Transpiling however, in this context, is the act of translating a program to another program, whilst maintaining a similar level of abstraction. An example can be to transpile a program written in Python to an equivalent program in Rust, for performance optimisation reasons \cite{DBLP:conf/samos/LunnikiviJ020}. \hfill \\

One of our output targets is pseudocode. We are not attempting to create a standardisation or ground truth for pseudocode, but rather we lean on the algorithm2e\footnote{https://www.ctan.org/pkg/algorithm2e} library, which is an aleady established environment for writing algorithms in \LaTeX. One thing we \textit{can} ensure, is consistency, removing the burden of "should this be in italics or boldface or not" from the author. In addition, the author can write her algorithm in an executable programming language, and when sure that it yields the desired result, she can safely use Psnodig to receive a \LaTeX file and a companion PDF of her work. \hfill \\

Other freely available alternatives for transpiling source code to pseudocode is Code Kindle\footnote{https://devpost.com/software/code-kindle} and Pseudogen, a tool introduced by Oda et. al\cite{DBLP:conf/kbse/OdaFNHSTN15}. Both solutions use statistical machine translation, which is a technique to train a model on previously translated and analyzed information and conversations\footnote{https://en.wikipedia.org/wiki/Statistical\_machine\_translation}. With Pseudogen, code is transpiled to purely natural language. Code Kindle's results are less verbose, though in most cases still a description accompanying the original source code. \hfill \\

The other target output of Psnodig is flowcharts. This is a representation very different to the original source code, and the main difficulty consists of keeping the level of abstraction. The main benefit of this representation, is to see the code from a completely different perspective. This can be refreshing when you have been trying to debug code for an extended period of time with no result [source?]. \hfill \\

Currently, to our knowledge, the only freely available tool solely dedicated to converting source code to flowcharts is Code2Flow\footnote{https://app.code2flow.com/}. This is a DSL with support for most common programming concepts like statements, loops, conditionals, and more. We are aware of other tools which also convert code to diagrams, like Mermaid.js\footnote{https://mermaid.js.org/} and Diagrams\footnote{https://diagrams.mingrammer.com/}, but they are either very general purpose, or focus on different types of diagrams. \hfill \\

The positive effect of teaching with flowcharts as an alternative to traditional code has been researched since the 1980's by Scanlan \cite{DBLP:journals/software/Scanlan89}, and also in recent times by Giordano et. al \cite{7096016}, yet direct translation from source code to flowcharts does not seem very widespread. \hfill \\

We spend much more time reading code than we do writing code\cite[14]{martin2008clean}, and tools like IDEs and linters can only help us so much long term when it is the logic of our programs that we fail to grasp. We believe that the Psnodig tool can be an alternative for both authors wanting to present their algorithms, as well as students wishing to get a better grasp of them.

% Mention MAINTAINABILITY. For once, doing this shit by hand means we must maintain all (e.g. source code, psueodocode and flowcharts) "by hand". But now we don't have to anymore!

\section{Research Questions}

In this thesis, we aim to answer the following research questions: \hfill \\

\begin{quote}
    \textbf{RQ1:} Can we transpile source code to pseudocode whilst maintaining a similar level of abstract?

    \textbf{RQ2:} Can we transpile source code to flowcharts whilst maintaining a similar level of abstract?

    \sout{\textbf{RQ3:} Can people in academia find such a tool to be helpful in better understanding small-to-medium sized code chunks?} Keep, replace or remove entirely?
\end{quote}

\section{Contributions}

The main contribution of this thesis is the Psnodig tool for transpiling executable source code to various presentation-only versions of said code, giving people in academia an easy and accessible way of looking at their code from a different angle. By using the Psnodig tool, people can spend more time writing code and less time mastering \LaTeX libraries, writing boilerplate and worrying about consistency. \hfill \\

The Psnodig tool is written entirely in the Haskell programming language\footnote{https://www.haskell.org/}, offering a syntax rich enough for writing all algorithms and data structures introduced in the introductory course to algorithms and data structures at the University of Oslo\footnote{https://www.uio.no/studier/emner/matnat/ifi/IN2010/index-eng.html}. The tool comes with a parser for the Gourmet programming language, as well as a writer for both pseudocode and flowcharts in \LaTeX, utilising the Algorithm2e package\footnote{https://www.ctan.org/pkg/algorithm2e} and the TikZ package\footnote{https://www.overleaf.com/learn/latex/TikZ\_package}. \hfill \\
% https://www.overleaf.com/learn/latex/LaTeX_Graphics_using_TikZ%3A_A_Tutorial_for_Beginners_(Part_3)%E2%80%94Creating_Flowcharts

To summarise, the contributions include:
\begin{itemize}
    \item Psnodig, a tool for transpiling code from one representation to another. Also comes with an interpreter which works on the intermediate AST representation.
    \item The Gourmet programming language, inspired by Go and Python, as a proof of concept. This includes a parser for converting tokens to an AST, as well as a writer to convert the AST back to Gourmet code.
    \item A writer for presenting an AST with text based pseudocode, utilising the Algorithm2e package in \LaTeX.
    \item A writer for presenting an AST with image based pseudocode (flowcharts), utilising the TikZ package, also in \LaTeX.
\end{itemize}

\section{Chapter Overview}

\textbf{Chapter 2} provides a clear definition of pseudocode which we carry with us for the remainder of the thesis, as well as some background on transpiling. \hfill \\

\textbf{Chapter 3} breaks down the exact problem we are looking at, as well as analysing selected tools which already offer some of the functionality we aim to contribute with. \hfill \\

\textbf{Chapter 4} delves further into how we intend to solve the problem we introduce in the previous chapter, and compares our tool to what we believe are the shortcomings of its competitors. \hfill \\

\textbf{Chapter 5} provides concrete implementation details of Psnodig. \hfill \\

\textbf{Chapter 6} covers how we evaluated Psnodig, how it really works in practice, strengths, weaknesses, and how we tried to make sure it actually works as intended. \hfill \\

\textbf{Chapter 7} discusses how the solution holds up agains the problem, and whether or not it fills the holes we believe exist in the alternatives on the market. \hfill \\

\textbf{Chapter 8} concludes the work of this thesis, discussing the research questions and future work.

\section{Project Source Code}

All the source code from the master thesis can be found on Github\footnote{https://github.com/dashboard} (NOTE: NÅ LIGGER DEN PÅ UIO ENTERPRISE-GITHUBEN. BURDE VÆRE MULIG Å OVERFØRE DEN TIL GITHUB.COM SLIK AT DEN FORBLIR TILGJENGELIG OGSÅ ETTER AT JEG LEVERER OPPGAVEN OG MISTER UIO-RETTIGHETENE :SMILEFJES:).