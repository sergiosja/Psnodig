\chapter{Introduction}

Pseudocode is commonly used to provide a description of an algorithm at a suitable level of abstraction. It is meant to work as a comprimise between a low-level implementation in a specific programming language, and a natural language description of a problem solution~\cite{whatIsPseudocode}. \\

An advantage of pseudocode is the lack of standardisation, therefore authors are not tied down to the syntax of any particular programming language. This gives them complete freedom to omit or de-emphasise certain aspects of their algorithms. Consequently, pseudocode is first and foremost aimed to serve as a tool for presentation. \\

Consequently, as pseudocode is not executable, there is no omniscient way to verify its correctness. This can, in turn, lead to accidental inclusion of critical inaccuracies. When working within an integrated development environment, most popular languages have support for static analysis tools that detect anti-patterns and warn about bad practices~\cite{manyLinters, whatIsALinter}. 
When writing pseudocode, on the other hand, we are left to our own devices, as identification of anti-patterns and bad practices generally relies on the existence of established standards. \\

\section{Motivation}

Correct presentations are important in education where the goal is to teach students concepts they were previously unfamiliar with. At university level, concepts within algorithms and data structures have traditionally proved challenging for undergraduates~\cite{algorithmsAreHard1, algorithmsAreHard2, algorithmsAreHard3}. If their first impression of an algorithm is an incorrect presentation, their path is already hampered. \\

In this thesis we present a tool called \textbf{Psnodig} (pronounced snoo-dee\footnote{Imagine the \textbf{sn} in \textbfit{snow}, the \textbf{oo} in \textbfit{cool}, and the \textbf{dee} in \textbfit{deep}. The silent \textbf{p} is an ode to the silent p in the word \textbfit{pseudocode}, naturally.}), which allows us to convert source code to other, less technical presentations. The presentation targets in this thesis are mainly pseudocode and flowcharts. The target audience is people who study or work with algorithms. \\

The positive effect of teaching algorithms with flowcharts as an alternative to traditional code has been researched since at least the 1980's, and is still being researched this decade~\cite{flowchartsAreGood1, flowchartsAreGood2, flowchartsAreGood3}. However, tools for direct conversion from source code to flowcharts does not seem to be widespread. \\

We spend much more time \textit{reading} code than we do writing it~\cite[14]{weReadMoreThanWeCode}. Tools like IDEs and linters can only help us so much when it is the \textit{logic} of our programs that we fail to grasp. We believe that Psnodig can be a valuable tool for authors who want to present their algorithms, as well as students who wish to get a better understanding of them. \\

We aim to promote algorithmic thinking through various forms of representation. We believe that this can aid us to better understand and modify our code, which in turn can lead to more efficient and effective programming practices. By not having to worry about syntactic intricacies, the audience can focus entirely on logic underlying the algorithms. \\

Another key point is that the software that \textit{do} let us convert source code to other representations, are completely independent of each other. Most of them also operate with their own domain-specific language to write source code, which makes them even more unavailable. Psnodig, on the other hand, attempts to centralise all these resources, making it a powerful all-in-one tool for any conversion we might wish for. \\

\section{Goals}

The overall goal of this thesis is to construct a tool with the following properties:

\begin{itemize}
    \item \textbf{Extensible}, users can add parsers and code generators to Psnodig.
    \item \textbf{Executable}, users can run the source code they have written.
    \item \textbf{Presentable}, users can use Psnodig to convert their source code to a different level of abstraction.
\end{itemize}

We will use selected algorithms from \cite{pseudocodeInBook2} to evaluate to which degree the tool posesses these properties in Chapter 6. We will then discuss them more in depth in Chapter 7.

\section{Contributions}

The main contribution of this thesis is the Psnodig tool, which facilitates the conversion of executable source code to a number of different formats. The tool gives people an easy and accessible way to look at their code from a different perspective. \\

By using the Psnodig tool, we aim to help people to spend more time writing meaningful code and less time mastering LaTeX libraries, writing boilerplate code, and worrying about maintaining multiple sources. \\

We are able to add parsers and writers to Psnodig at will, making it a flexible tool. Psnodig is materialised through the following implementations:

\begin{itemize}
    \item A parser for an imperative, C-like programming language we have coined \textbf{Gourmet}.
    \item Two writers converting Psnodig ASTs to pseudocode and flowcharts.
    \item Two writers converting Psnodig ASTs to C-like programs and Python-like programs.
\end{itemize}

To the best of our knowledge, Psnodig is the first tool of its kind for transpiling source code to these formats, in additional to being openly available and extensible.

%---

%The Psnodig tool comes with a parser for an imperative C-like language we call Gourmet, to serve as a proof of concept. It is also accompanied by four code generators that can convert an internal representation of Psnodig to four different target programs. They are Gourmet itself, a subset of Python, pseudocode and lastly flowcharts. The two latter are presented in LaTeX, and utilise the Algorithm2e- and TikZ packages, respectively. \\

%To summarise, the contributions include:
%\begin{itemize}
%    \item Psnodig, a tool for converting code from one representation to another. It comes with an interpreter that works on the internal representation.
%    \item A parser and writer for a simple C-like language we have coined \texttt{Gourmet}.
%    \item A writer for a subset of Python, that we have coined \texttt{Pytite}.
%    \item A writer for presenting ASTs with text based pseudocode, utilising the Algorithm2e package in LaTeX.
%    \item A writer for presenting ASTs with flowcharts, utilising the TikZ package, also in LaTeX.
%\end{itemize}

\section{Project Source Code}

All the source code from the master thesis can be found on Github.\footnote{More specifically, the entire thesis work can be found at \url{https://github.com/sergiosja/Master}. It is actually divided into two folders: \textbfit{psnodig}, containing the source code for the Psnodig tool, and \textbfit{thesis}, containing the LaTeX source code for this thesis.}

\forsup{NOTE: Må huske å gjøre repoet public!}